% Please do not change the document class
\documentclass{scrartcl}

% Please do not change these packages
\usepackage[hidelinks]{hyperref}
\usepackage[none]{hyphenat}
\usepackage{setspace}
\doublespace

% You may add additional packages here
\usepackage{amsmath}
\usepackage{graphicx}
\usepackage{wrapfig}
\graphicspath{ {./images/} }

% Please include a clear, concise, and descriptive title
\title{HCI FOR VR \& AR} 

% Please do not change the subtitle
\subtitle{COMP210- Research Journal}

% Please put your student number in the author field
\author{1507516}

\begin{document}

\maketitle

\abstract{}


\section{Introduction}
This paper is focused on research into Augmented and Virtual Reality techniques that could be applied to games.

\cite{nielsen1990heuristic}

\section {Using Laser Projectors for Augmented Reality\cite{Schwerdtfeger:2008}}

This paper explores the idea of using laser projectors as an alternative to Head Mounted Displays (HMD).
The paper focuses on setting up laser projectors for industrial Augmented Reality applications \cite{Schwerdtfeger:2008}.

In one of the proposed scenarios that this AR system could work well is for maintenance of complex tools and machines. For example the laser projector would project and highlight points on the tools where screws need to be fitted. This could make for some interesting mechanics if implemented into an AR game, for example the lasers can highlight objects in front of the player that they have to interact with in a sequence.

\section {Possession Techniques for Interaction in Real-time Strategy Augmented Reality Games \cite{Phillips:2005}}
This paper covers interaction techniques used in Augmented Reality (AR) for Real Time Stategy (RTS) games.

This paper introduces a new technique called ``possession'' which attempts to allow the player to manage a large force of RTS units without the user being confined to how fast they can move in the real world.

What possession does is it allows the player the ability to move inside the head of any of their units,  and manage their forces from within that unit. 

This technique is specific to the game they are calling ``ARBattleCommander'' which is an outdoor AR strategy game.

\section{A virtual reality-based multi-player game using fine-grained localization \cite{de2015virtual}}
This paper presents a mobile framework where the player can move around freely while wearing a Head Mounted Display (HMD). 

The paper analyses multiple approaches for localizing the player and builds a suitable localization method that tracks the players movements in the virtual world. 
One approach for localizing the player they used image processing and the OpenCV library to track the players position with cameras. Another approach they looked into was using Bluetooth Low Energy (BLE) to track the players position in world space. 

\section{An Ant's Life: Storytelling in Virtual Reality \cite{Leo:2015}}
This paper takes a new approach of integrating 2D art into a 3D environment. This VR game aims to try and bridge the virtual and real world to make VR games more accessible to a wider audience.
They have designed the game in unity for the Oculus Rift.

They aimed to have a unique look to their game as the player was in the point of view of an ant, and the sprites were 2D, but in a 3D world. 

To try and immerse the player more they used physical props that were designed around the aesthetic of the game, i.e. the player stood on a giant leaf and had red stuffed cherries to carry during the game.

\section{ Multi-player VR Game Built Upon Wireless Sensor Network \cite{Jee:2008}}
This paper presents a multi-player VR game that uses motion-tracking with a wireless sensor network to allow multiple players to play the VR game together in the real world.
This paper is based of a previous paper by \cite{Eom:2006} which proposes the VR game platform that uses \textit{Magic Wands} to measure the players movements.

\bibliographystyle{ieeetr}
\bibliography{comp210_Research_Journal}

\end{document}
